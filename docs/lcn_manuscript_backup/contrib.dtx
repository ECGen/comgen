% \iffalse meta-comment
%
% This program can be distributed and/or modified under the terms
% of the LaTeX Project Public License either version 1.3c of this
% license or (at your option) any later version.
% The latest version of this license is in
%    http://www.latex-project.org/lppl.txt
% and version 1.3c or later is part of all distributions of LaTeX
% version 2005/12/01 or later.
%
% This file has the LPPL maintenance status "maintained".
%
% \fi
%
% \iffalse
%<*opt>
% \fi
%    \begin{macrocode}
%% This is an example of config file *for syntax only*.
%%
%% Copy it to your document's directory and add
%% \renewcommand{\DefaultOptionsFile}{optfile.cfl}
%% to your document's preamble to use it.
%%
%% The settings below are just *examples*, they are not meant to be good!
%% Proper values heavily depend on the font used!
%%
%% Letters not listed below, will get either the default parameters,
%% or those given as optional argument of \lettrine, if any.
%%
%% The values of the parameters set in this file can be overridden locally
%% using the optional argument of \lettrine.
%%
%% A
\LettrineOptionsFor{A}{slope=0.1\LettrineWidth, findent=-.5em, nindent=.7em}
%% Same parameters for \`A
\LettrineOptionsFor{\`A}{slope=0.1\LettrineWidth, findent=-.5em, nindent=0.7em}
%% C'
\LettrineOptionsFor{C'}{nindent=-0.5em}
%% I and J hang into the margin
\LettrineOptionsFor{I}{lhang=1, nindent=0pt}
\LettrineOptionsFor{J}{lhang=1, nindent=0pt}
%% Q
\LettrineOptionsFor{Q}{loversize=-0.1, lraise=0.1}
%% V
\LettrineOptionsFor{V}{slope=-0.1\LettrineWidth, lhang=0.5, nindent=0pt}
%    \end{macrocode}
% \iffalse
%</opt>
% \fi
%
% \iffalse
%<*pzc2>
% \fi
%    \begin{macrocode}
%%% This is file `pzc2.cfl'.
%%% It is provided under the LPPL. Copyright is held by Kevin M. Dunn.
%%% If you modify this file you *MUST* change its name.

%%% This file contains config values to simplify the use of the
%%% Zapf-Chancery (aka pzc) font with the lettrine package by Daniel Flipo.
%%% This file provides the parameters for 2 DefaultLines.

%%% If you want to use this file, your document should contain
%%% the next three lines, without the leading percent signs.
%%% \setcounter{DefaultLines}{2}
%%% \renewcommand\LettrineFontHook{\fontencoding{T1}\fontfamily{pzc}\selectfont}
%%% \renewcommand{\DefaultOptionsFile}{pzc2.cfl}

%%% Comments about this file can be sent to Kevin Dunn
%%% (cavema2 at cavemanchemistry.com)

\ProvidesFile{pzc2.cfl}[2006/03/19 v0.1 Kevin M. Dunn]

\LettrineOptionsFor{A}{loversize= 0.35,lraise= 0.00,lhang= 0.35,findent= 0.05\LettrineWidth,nindent= 0.25\LettrineWidth}
\LettrineOptionsFor{B}{loversize= 0.35,lraise= 0.00,lhang= 0.35,findent= 0.10\LettrineWidth,nindent= 0.08\LettrineWidth}
\LettrineOptionsFor{C}{loversize= 0.35,lraise= 0.10,lhang= 0.35,findent= 0.00\LettrineWidth,nindent= 0.28\LettrineWidth}
\LettrineOptionsFor{D}{loversize= 0.35,lraise= 0.00,lhang= 0.35,findent= 0.15\LettrineWidth,nindent= 0.05\LettrineWidth}
\LettrineOptionsFor{E}{loversize= 0.35,lraise= 0.00,lhang= 0.35,findent= 0.00\LettrineWidth,nindent= 0.28\LettrineWidth}
\LettrineOptionsFor{F}{loversize= 0.35,lraise= 0.00,lhang= 0.35,findent= 0.10\LettrineWidth,nindent= 0.00\LettrineWidth}
\LettrineOptionsFor{G}{loversize= 0.35,lraise= 0.10,lhang= 0.35,findent= 0.10\LettrineWidth,nindent= 0.17\LettrineWidth}
\LettrineOptionsFor{H}{loversize= 0.35,lraise= 0.00,lhang= 0.35,findent= 0.10\LettrineWidth,nindent= 0.20\LettrineWidth}
\LettrineOptionsFor{I}{loversize= 0.35,lraise= 0.00,lhang= 0.35,findent= 0.05\LettrineWidth,nindent= 0.25\LettrineWidth}
\LettrineOptionsFor{J}{loversize= 0.35,lraise= 0.00,lhang= 0.35,findent= 0.15\LettrineWidth,nindent= 0.00\LettrineWidth}
\LettrineOptionsFor{K}{loversize= 0.35,lraise= 0.00,lhang= 0.35,findent= 0.05\LettrineWidth,nindent= 0.13\LettrineWidth}
\LettrineOptionsFor{L}{loversize= 0.35,lraise= 0.00,lhang= 0.35,findent=-0.02\LettrineWidth,nindent= 0.30\LettrineWidth}
\LettrineOptionsFor{M}{loversize= 0.35,lraise= 0.00,lhang= 0.35,findent= 0.05\LettrineWidth,nindent= 0.15\LettrineWidth}
\LettrineOptionsFor{N}{loversize= 0.35,lraise= 0.00,lhang= 0.35,findent= 0.10\LettrineWidth,nindent= 0.10\LettrineWidth}
\LettrineOptionsFor{O}{loversize= 0.35,lraise= 0.00,lhang= 0.35,findent= 0.15\LettrineWidth,nindent= 0.08\LettrineWidth}
\LettrineOptionsFor{P}{loversize= 0.35,lraise= 0.00,lhang= 0.35,findent= 0.25\LettrineWidth,nindent= 0.05\LettrineWidth}
\LettrineOptionsFor{Q}{loversize= 0.35,lraise= 0.00,lhang= 0.35,findent= 0.15\LettrineWidth,nindent= 0.10\LettrineWidth}
\LettrineOptionsFor{R}{loversize= 0.35,lraise= 0.00,lhang= 0.35,findent= 0.08\LettrineWidth,nindent= 0.15\LettrineWidth}
\LettrineOptionsFor{S}{loversize= 0.35,lraise= 0.00,lhang= 0.35,findent= 0.10\LettrineWidth,nindent= 0.13\LettrineWidth}
\LettrineOptionsFor{T}{loversize= 0.35,lraise= 0.00,lhang= 0.35,findent= 0.05\LettrineWidth,nindent= 0.20\LettrineWidth}
\LettrineOptionsFor{U}{loversize= 0.35,lraise= 0.00,lhang= 0.35,findent= 0.05\LettrineWidth,nindent= 0.15\LettrineWidth}
\LettrineOptionsFor{V}{loversize= 0.35,lraise= 0.00,lhang= 0.35,findent= 0.18\LettrineWidth,nindent=-0.15\LettrineWidth}
\LettrineOptionsFor{W}{loversize= 0.35,lraise= 0.00,lhang= 0.35,findent= 0.15\LettrineWidth,nindent=-0.05\LettrineWidth}
\LettrineOptionsFor{X}{loversize= 0.35,lraise= 0.00,lhang= 0.35,findent= 0.00\LettrineWidth,nindent= 0.35\LettrineWidth}
\LettrineOptionsFor{Y}{loversize= 0.35,lraise= 0.00,lhang= 0.35,findent= 0.10\LettrineWidth,nindent= 0.05\LettrineWidth}
\LettrineOptionsFor{Z}{loversize= 0.35,lraise= 0.00,lhang= 0.35,findent= 0.00\LettrineWidth,nindent= 0.30\LettrineWidth}
%    \end{macrocode}
% \iffalse
%</pzc2>
% \fi
%
% \iffalse
%<*pzc3>
% \fi
%    \begin{macrocode}
%%% This is file `pzc3.cfl'.
%%% It is provided under the LPPL. Copyright is held by Kevin M. Dunn.
%%% If you modify this file you *MUST* change its name.

%%% This file contains config values to simplify the use of the
%%% Zapf-Chancery (aka pzc) font with the lettrine package by Daniel Flipo.
%%% This file provides the parameters for 3 DefaultLines.

%%% If you want to use this file, your document should contain
%%% the next three lines, without the leading percent signs.
%%% \setcounter{DefaultLines}{3}
%%% \renewcommand\LettrineFontHook{\fontencoding{T1}\fontfamily{pzc}\selectfont}
%%% \renewcommand{\DefaultOptionsFile}{pzc3.cfl}

%%% Comments about this file can be sent to Kevin Dunn
%%% (cavema2 at cavemanchemistry.com)

\ProvidesFile{pzc3.cfl}[2006/03/19 v0.1 Kevin M. Dunn]

\LettrineOptionsFor{A}{loversize= 0.35,lraise= 0.00,lhang= 0.35,findent=0.05\LettrineWidth,nindent= 0.25\LettrineWidth}
\LettrineOptionsFor{B}{loversize= 0.35,lraise= 0.00,lhang= 0.35,findent=0.10\LettrineWidth,nindent= 0.08\LettrineWidth,slope=-0.08\LettrineWidth}
\LettrineOptionsFor{C}{loversize= 0.35,lraise= 0.10,lhang= 0.35,findent=0.00\LettrineWidth,nindent= 0.28\LettrineWidth}
\LettrineOptionsFor{D}{loversize= 0.35,lraise= 0.00,lhang= 0.35,findent=0.15\LettrineWidth,nindent= 0.05\LettrineWidth,slope=-0.05\LettrineWidth}
\LettrineOptionsFor{E}{loversize= 0.35,lraise= 0.00,lhang= 0.35,findent=0.00\LettrineWidth,nindent= 0.28\LettrineWidth}
\LettrineOptionsFor{F}{loversize= 0.35,lraise= 0.05,lhang= 0.35,findent=0.10\LettrineWidth,nindent= 0.00\LettrineWidth}
\LettrineOptionsFor{G}{loversize= 0.30,lraise= 0.30,lhang= 0.5,findent=0.10\LettrineWidth,nindent= 0.15\LettrineWidth}
\LettrineOptionsFor{H}{loversize= 0.35,lraise= 0.00,lhang= 0.35,findent=0.10\LettrineWidth,nindent= 0.20\LettrineWidth}
\LettrineOptionsFor{I}{loversize= 0.35,lraise= 0.00,lhang= 0.35,findent=0.05\LettrineWidth,nindent= 0.25\LettrineWidth}
\LettrineOptionsFor{J}{loversize= 0.35,lraise= 0.05,lhang= 0.35,findent=0.15\LettrineWidth,nindent= 0.00\LettrineWidth}
\LettrineOptionsFor{K}{loversize= 0.30,lraise= 0.20,lhang= 0.35,findent=0.05\LettrineWidth,nindent= 0.13\LettrineWidth,slope= 0.20\LettrineWidth}
\LettrineOptionsFor{L}{loversize= 0.35,lraise= 0.00,lhang=0.35,findent=-0.02\LettrineWidth,nindent= 0.30\LettrineWidth}
\LettrineOptionsFor{M}{loversize= 0.35,lraise= 0.00,lhang= 0.35,findent=0.05\LettrineWidth,nindent= 0.15\LettrineWidth}
\LettrineOptionsFor{N}{loversize= 0.30,lraise= 0.20,lhang= 0.35,findent=0.10\LettrineWidth,nindent= 0.10\LettrineWidth,slope= 0.20\LettrineWidth}
\LettrineOptionsFor{O}{loversize= 0.35,lraise= 0.00,lhang= 0.35,findent=0.15\LettrineWidth,nindent= 0.08\LettrineWidth}
\LettrineOptionsFor{P}{loversize= 0.35,lraise= 0.00,lhang= 0.35,findent=0.25\LettrineWidth,nindent= 0.05\LettrineWidth}
\LettrineOptionsFor{Q}{loversize= 0.30,lraise= 0.25,lhang= 0.35,findent=0.15\LettrineWidth,nindent= 0.10\LettrineWidth,slope= 0.20\LettrineWidth}
\LettrineOptionsFor{R}{loversize= 0.30,lraise= 0.20,lhang= 0.35,findent=0.08\LettrineWidth,nindent= 0.15\LettrineWidth,slope= 0.20\LettrineWidth}
\LettrineOptionsFor{S}{loversize= 0.35,lraise= 0.00,lhang= 0.35,findent=0.10\LettrineWidth,nindent= 0.13\LettrineWidth}
\LettrineOptionsFor{T}{loversize= 0.35,lraise= 0.00,lhang= 0.35,findent=0.05\LettrineWidth,nindent= 0.20\LettrineWidth}
\LettrineOptionsFor{U}{loversize= 0.35,lraise= 0.00,lhang= 0.35,findent=0.05\LettrineWidth,nindent= 0.15\LettrineWidth}
\LettrineOptionsFor{V}{loversize= 0.35,lraise= 0.00,lhang= 0.35,findent=0.18\LettrineWidth,nindent=-0.05\LettrineWidth,slope=-0.15\LettrineWidth}
\LettrineOptionsFor{W}{loversize= 0.35,lraise= 0.00,lhang= 0.35,findent=0.18\LettrineWidth,nindent=-0.05\LettrineWidth,slope=-0.10\LettrineWidth}
\LettrineOptionsFor{X}{loversize= 0.35,lraise= 0.00,lhang= 0.35,findent=0.00\LettrineWidth,nindent= 0.35\LettrineWidth}
\LettrineOptionsFor{Y}{loversize= 0.30,lraise= 0.10,lhang= 0.40,findent=0.10\LettrineWidth,nindent= 0.00\LettrineWidth,slope=-0.10\LettrineWidth}
\LettrineOptionsFor{Z}{loversize= 0.35,lraise= 0.00,lhang= 0.35,findent=0.00\LettrineWidth,nindent= 0.30\LettrineWidth}
%    \end{macrocode}
% \iffalse
%</pzc3>
% \fi
%
% \iffalse
%<*pacl>
% \fi
%    \begin{macrocode}
%%% This is file `pacl.cfl'.
%%% It is provided under the LPPL. Copyright is hold by Pascal Kockaert.
%%% If you modify this file you *MUST* change its name.

%%% This file contains config values to simplify the use of the
%%% ACaslon-SwashItalic (aka pacri8s) font from Adobe
%%% with the lettrine package by Daniel Flipo.

%%% If you want to use this file, your document should contain
%%% the two next lines, without the leading percent signs.
%%% \renewcommand\LettrineFontHook{\fontencoding{T1}\fontfamily{pacl}\selectfont}
%%% \renewcommand{\DefaultOptionsFile}{pacl.cfl}

%%% The fontfamily pacl is defined through the file T1pacl.fd.
%%% The contents of T1pacl.fd is listed at the end of this file.

%%% The settings below were defined according to my visual tastes.
%%% No mathematical rule based on the metrices was applied.
%%% Though the result may not please you, it should be better
%%% than the default placement. This said, comments are welcome.

%%% Comments about this file can be sent to Pascal.Kockaert
%%% at the mail server ulb.ac.be.

\ProvidesFile{pacl.cfl}[2003/08/24 v0.1 Pascal Kockaert]

\LettrineOptionsFor{A}{loversize= 0.15,lraise= 0.02,lhang= 0.30,findent= 0.00\LettrineWidth,nindent= 0.50em}
\LettrineOptionsFor{B}{loversize= 0.15,lraise= 0.00,lhang= 0.25,findent=-0.05\LettrineWidth,nindent= 0.50em}
\LettrineOptionsFor{C}{loversize=-0.10,lraise= 0.25,lhang= 0.05,findent= 0.00\LettrineWidth,nindent= 0.50em}
\LettrineOptionsFor{D}{loversize= 0.15,lraise= 0.00,lhang= 0.20,findent=-0.07\LettrineWidth,nindent= 0.50em}
\LettrineOptionsFor{E}{loversize= 0.15,lraise= 0.00,lhang= 0.05,findent=-0.10\LettrineWidth,nindent= 0.50em}
\LettrineOptionsFor{F}{loversize=-0.10,lraise= 0.20,lhang= 0.00,findent= 0.05\LettrineWidth,nindent= 0.50em}
\LettrineOptionsFor{G}{loversize=-0.10,lraise= 0.20,lhang= 0.05,findent=-0.05\LettrineWidth,nindent= 0.50em}
\LettrineOptionsFor{H}{loversize= 0.05,lraise= 0.00,lhang= 0.25,findent= 0.00\LettrineWidth,nindent= 0.50em}
\LettrineOptionsFor{I}{loversize= 0.15,lraise= 0.00,lhang= 0.00,findent= 0.00\LettrineWidth,nindent= 0.50em}
\LettrineOptionsFor{J}{loversize=-0.10,lraise= 0.20,lhang= 0.00,findent= 0.00\LettrineWidth,nindent= 0.50em}
\LettrineOptionsFor{K}{loversize=-0.10,lraise= 0.25,lhang= 0.00,findent= 0.00\LettrineWidth,nindent= 0.45\LettrineWidth}
\LettrineOptionsFor{L}{loversize= 0.00,lraise= 0.20,lhang= 0.00,findent=-0.10\LettrineWidth,nindent= 0.40\LettrineWidth}
\LettrineOptionsFor{M}{loversize= 0.15,lraise= 0.00,lhang= 0.24,findent=-0.05\LettrineWidth,nindent= 0.10\LettrineWidth}
\LettrineOptionsFor{N}{loversize=-0.05,lraise= 0.25,lhang= 0.13,findent= 0.05\LettrineWidth,nindent= 0.35\LettrineWidth}
\LettrineOptionsFor{O}{loversize= 0.15,lraise= 0.00,lhang= 0.05,findent=-0.12\LettrineWidth,nindent= 0.15\LettrineWidth}
\LettrineOptionsFor{P}{loversize= 0.15,lraise= 0.00,lhang= 0.25,findent= 0.00\LettrineWidth,nindent= 0.50em}
\LettrineOptionsFor{Q}{loversize=-0.15,lraise= 0.25,lhang= 0.10,findent=-0.05\LettrineWidth,nindent= 0.55\LettrineWidth}
\LettrineOptionsFor{R}{loversize=-0.15,lraise= 0.25,lhang= 0.20,findent=-0.05\LettrineWidth,nindent= 0.45\LettrineWidth}
\LettrineOptionsFor{S}{loversize=-0.10,lraise= 0.25,lhang=-0.05,findent=-0.05\LettrineWidth,nindent= 0.50em}
\LettrineOptionsFor{T}{loversize= 0.15,lraise= 0.00,lhang= 0.15,findent= 0.00\LettrineWidth,nindent= 0.15\LettrineWidth}
\LettrineOptionsFor{U}{loversize= 0.15,lraise= 0.00,lhang= 0.05,findent= 0.00\LettrineWidth,nindent= 0.50em}
\LettrineOptionsFor{V}{loversize= 0.15,lraise= 0.00,lhang= 0.20,findent=-0.05\LettrineWidth,nindent= 0.50em}
\LettrineOptionsFor{W}{loversize= 0.15,lraise= 0.00,lhang= 0.15,findent=-0.05\LettrineWidth,nindent= 0.50em}
\LettrineOptionsFor{X}{loversize= 0.15,lraise= 0.00,lhang= 0.45,findent= 0.05\LettrineWidth,nindent= 0.15\LettrineWidth}
\LettrineOptionsFor{Y}{loversize= 0.15,lraise= 0.00,lhang= 0.05,findent= 0.20\LettrineWidth,nindent= 0.25\LettrineWidth}
\LettrineOptionsFor{Z}{loversize= 0.05,lraise= 0.15,lhang= 0.15,findent= 0.00\LettrineWidth,nindent= 0.30\LettrineWidth}
\let\EOF\endinput
\EOF

%% The installation of the pacl family can be performed using the
%% fontinst package.
%% You must own the font ACaslon-SwashItalic, that is an AFM and a PFB file
%% which should be renamed as padri8w.afm and padri8w.pfb.

%% You should process the file Makepacl.tex (see below) through TeX,
%% and follow the instructions of the fontinst manual to finish the install.
%% The file T1pacl.fd should be defined as below and put with other local
%% FD files.

%%% File Makepacl.tex
\input fontinst.sty
\installfonts
   \declareencoding{T1-SWASH}{T1}
   \fromafm{pacri8s} %%% File containing metrics of ACaslon-SwashItalic
   \installfont{pacri9s}{pacri8s}{T1}{T1}{pacl}{m}{n}{}
\endinstallfonts
\bye
%%% End of file Makepacl.tex

%%% The pacl family is defined by the file T1pacl.fd, as follows

%%% File T1pacl.fd
%%% THIS FILE SHOULD BE PUT IN A TEX INPUTS DIRECTORY
\ProvidesFile{t1pacl.fd}[2003/08/24 v1.0 Pascal Kockaert]
\DeclareFontFamily{T1}{pacl}{}
\DeclareFontShape{T1}{pacl}{m}{n}{<->pacri9s}{}
%%% End of file T1pacl.fd
%    \end{macrocode}
% \iffalse
%</pacl>
% \fi
%
% \iffalse
%<*padl>
% \fi
%    \begin{macrocode}
%%% This is file `padl.cfl'.
%%% It is provided under the LPPL. Copyright is hold by Pascal Kockaert.
%%% If you modify this file you *MUST* change its name.

%%% This file contains config values to simplify the use of the
%%% AGaramondAlt-Italic (aka padri8w) font from Adobe
%%% with the lettrine package by Daniel Flipo.

%%% If you want to use this file, your document should contain
%%% the two next lines, without the leading percent signs.
%%% \renewcommand\LettrineFontHook{\fontencoding{T1}\fontfamily{padl}\selectfont}
%%% \renewcommand{\DefaultOptionsFile}{padl.cfl}

%%% The fontfamily padl is defined through the file T1padl.fd.
%%% The contents of T1padl.fd is listed at the end of this file.

%%% The settings below were defined according to my visual tastes.
%%% No mathematical rule based on the metrices was applied.
%%% Though the result may not please you, it should be better
%%% than the default placement. This said, comments are welcome.

%%% Comments about this file can be sent to Pascal.Kockaert
%%% at the mail server ulb.ac.be.

\ProvidesFile{padl.cfl}[2003/08/24 v1.0 Pascal Kockaert]

\LettrineOptionsFor{A}{loversize= 0.05,lraise= 0.10,lhang= 0.40,findent= 0.000\LettrineWidth,nindent= 0.50em}
\LettrineOptionsFor{B}{loversize= 0.10,lraise= 0.00,lhang= 0.30,findent= 0.025\LettrineWidth,nindent= 0.50em}
\LettrineOptionsFor{C}{loversize= 0.00,lraise= 0.10,lhang= 0.15,findent= 0.175\LettrineWidth,nindent= 0.50em}
\LettrineOptionsFor{D}{loversize= 0.10,lraise= 0.00,lhang= 0.30,findent=-0.025\LettrineWidth,nindent= 0.50em}
\LettrineOptionsFor{E}{loversize= 0.10,lraise= 0.00,lhang= 0.10,findent= 0.100\LettrineWidth,nindent= 0.50em}
\LettrineOptionsFor{F}{loversize= 0.10,lraise= 0.00,lhang= 0.30,findent= 0.100\LettrineWidth,nindent= 0.50em}
\LettrineOptionsFor{G}{loversize=-0.08,lraise= 0.20,lhang= 0.10,findent= 0.000\LettrineWidth,nindent= 0.50em}
\LettrineOptionsFor{H}{loversize=-0.01,lraise= 0.00,lhang= 0.00,findent= 0.040\LettrineWidth,nindent= 0.50em}
\LettrineOptionsFor{I}{loversize= 0.10,lraise= 0.00,lhang= 0.00,findent= 0.050\LettrineWidth,nindent= 0.50em}
\LettrineOptionsFor{J}{loversize=-0.08,lraise= 0.20,lhang= 0.00,findent=-0.025\LettrineWidth,nindent= 0.50em}
\LettrineOptionsFor{K}{loversize=-0.05,lraise= 0.25,lhang= 0.00,findent=-0.050\LettrineWidth,nindent= 0.60\LettrineWidth}
\LettrineOptionsFor{L}{loversize= 0.00,lraise= 0.20,lhang= 0.00,findent= 0.100\LettrineWidth,nindent= 0.40\LettrineWidth}
\LettrineOptionsFor{M}{loversize= 0.10,lraise= 0.00,lhang= 0.40,findent= 0.040\LettrineWidth,nindent= 0.10\LettrineWidth}
\LettrineOptionsFor{N}{loversize= 0.00,lraise= 0.20,lhang= 0.20,findent=-0.010\LettrineWidth,nindent= 0.40\LettrineWidth}
\LettrineOptionsFor{O}{loversize= 0.10,lraise= 0.00,lhang= 0.20,findent=-0.010\LettrineWidth,nindent= 0.50em}
\LettrineOptionsFor{P}{loversize= 0.10,lraise= 0.00,lhang= 0.30,findent= 0.050\LettrineWidth,nindent= 0.50em}
\LettrineOptionsFor{Q}{loversize= 0.00,lraise= 0.20,lhang= 0.20,findent= 0.050\LettrineWidth,nindent= 0.70\LettrineWidth}
\LettrineOptionsFor{R}{loversize=-0.05,lraise= 0.25,lhang= 0.25,nindent= 0.500\LettrineWidth,nindent= 0.50\LettrineWidth}
\LettrineOptionsFor{S}{loversize= 0.10,lraise= 0.05,lhang= 0.00,findent= 0.000\LettrineWidth,nindent= 0.50em}
\LettrineOptionsFor{T}{loversize= 0.10,lraise= 0.00,lhang= 0.35,findent= 0.100\LettrineWidth,nindent= 0.50em}
\LettrineOptionsFor{U}{loversize= 0.10,lraise= 0.00,lhang= 0.20,findent= 0.050\LettrineWidth,nindent= 0.50em}
\LettrineOptionsFor{V}{loversize= 0.10,lraise= 0.00,lhang= 0.40,findent= 0.050\LettrineWidth,nindent= 0.50em}
\LettrineOptionsFor{W}{loversize= 0.10,lraise= 0.00,lhang= 0.30,findent= 0.040\LettrineWidth,nindent= 0.50em}
\LettrineOptionsFor{X}{loversize= 0.10,lraise= 0.00,lhang= 0.40,findent= 0.040\LettrineWidth,nindent= 0.20\LettrineWidth}
\LettrineOptionsFor{Y}{loversize= 0.10,lraise= 0.00,lhang= 0.20,findent= 0.100\LettrineWidth,nindent= 0.20\LettrineWidth}
\LettrineOptionsFor{Z}{loversize=-0.05,lraise= 0.25,lhang= 0.00,findent= 0.050\LettrineWidth,nindent= 0.55\LettrineWidth}
\let\EOF\endinput
\EOF

%% The installation of the padl family can be performed using the
%% fontinst package.
%% You must own the font AGaramondAlt-Italic, that is an AFM and a PFB
%% file which should be renamed as padri8w.afm and padri8w.pfb.

%% You should process the file Makepadl.tex (see below) through TeX,
%% and follow the instructions of the fontinst manual to finish the
%% install.
%% The file T1padl.fd should be defined as below and put with other
%% local FD files.

%%% File Makepadl.tex
\input fontinst.sty
\installfonts
   \declareencoding{T1-SWASH}{T1}
   \fromafm{padri8w} %%% File containing metrics of AGaramondAlt-Italic
   \gdef\uc#1#2{#1swash}
   \installfont{padw9w}{padri8w}{T1}{T1}{padl}{m}{n}{}
\endinstallfonts
\bye
%%% End of file Makepadl.tex

%% The padl family is defined by the file T1padl.fd, as follows

%%% File T1padl.fd
%%% THIS FILE SHOULD BE PUT IN A TEX INPUTS DIRECTORY
\ProvidesFile{t1padl.fd}[2003/08/24 v0.1 Pascal Kockaert]
\DeclareFontFamily{T1}{padl}{}
\DeclareFontShape{T1}{padl}{m}{n}{<->padw9w}{}
%%% End of file T1padl.fd
%    \end{macrocode}
% \iffalse
%</padl>
% \fi
%
% \iffalse
%<*dtx>
% \fi
%%
%% \CharacterTable
%%  {Upper-case    \A\B\C\D\E\F\G\H\I\J\K\L\M\N\O\P\Q\R\S\T\U\V\W\X\Y\Z
%%   Lower-case    \a\b\c\d\e\f\g\h\i\j\k\l\m\n\o\p\q\r\s\t\u\v\w\x\y\z
%%   Digits        \0\1\2\3\4\5\6\7\8\9
%%   Exclamation   \!     Double quote  \"     Hash (number) \#
%%   Dollar        \$     Percent       \%     Ampersand     \&
%%   Acute accent  \'     Left paren    \(     Right paren   \)
%%   Asterisk      \*     Plus          \+     Comma         \,
%%   Minus         \-     Point         \.     Solidus       \/
%%   Colon         \:     Semicolon     \;     Less than     \<
%%   Equals        \=     Greater than  \>     Question mark \?
%%   Commercial at \@     Left bracket  \[     Backslash     \\
%%   Right bracket \]     Circumflex    \^     Underscore    \_
%%   Grave accent  \`     Left brace    \{     Vertical bar  \|
%%   Right brace   \}     Tilde         \~}
%%
% \iffalse
%</dtx>
% \fi
%
% \Finale
\endinput

%%% Local Variables:
%%% fill-column: 70
%%% coding: utf-8
%%% End:
